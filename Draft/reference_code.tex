\section{Reference Code}
\label{sec:reference_code}

A reference implementation in Python code is provided in
\url{https://github.com/nucleosynthesis/SL-paper}.  This provides functions to
calculate the SL $a_I$, $b_I$, $c_I$, and $\rho_{IJ}$ coefficients, and an
\texttt{SLParams} class which automatically computes these and can use them to
compute profile and marginal likelihoods, log likelihood-ratios, and test
statistics. For convergence efficiency, the profile likelihood computation makes
use of the gradients of the SL log-likelihood with respect to the signal
strength $\mu$ and nuisance parameters $\bm{\theta}$, which we reproduce here to
assist independent implementations:
%
\begin{align}
  \ln \big( L(\bm{\alpha},\bm{\theta} )\pi(\bm{\theta}) \big) &=
  \sum_I^P \ln \mathrm{Poiss}(o_I|\bm{\alpha},\theta_I) - \frac{1}{2} \bm{\theta}^\mathrm{T} \bm{\rho}^{-1} \bm{\theta} - \frac{P}{2} \ln 2\pi \\
  %
  \Rightarrow \qquad
  \frac{\mathrm{d}\ln L}{\mathrm{d}\mu} &= \sum_I^P \left( \frac{o_I}{n_I(\theta_I)} - 1 \right) \cdot s_I(\bm{\alpha}) \\
  %
  \frac{\mathrm{d}\ln L}{\mathrm{d}\theta_i} &= \left( \frac{o_i}{n_i(\theta_i)} - 1 \right) \cdot \big( b_i + 2 c_i \theta_i \big) - \sum_I^P \rho_{iI}^{-1} \theta_I
\label{eq:SL_LHC}
\end{align}
